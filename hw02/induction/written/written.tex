\documentclass[11pt]{article}

\usepackage{amssymb, amsmath, amsthm, amsfonts}
\usepackage{listings,xcolor,lstautogobble}
\usepackage{fancyhdr,parskip}
\usepackage[margin=0.8in]{geometry}

% homework header
\pagestyle{fancy}
\setlength{\headheight}{0.3in}
\lhead{Induction and Recursion}
\rhead{15-150, Principles of Functional Programming}

% task environment
\newcommand{\task}[1]{\clearpage\textbf{Task #1}. \\[0.5em]}

% symbols
\newcommand{\eeq}{\ensuremath{\cong}}
\newcommand{\neeq}{\ensuremath{\ncong}}
\newcommand{\stepsTo}{\Longrightarrow}
\newcommand{\stepsToIn}[1]{\Longrightarrow^{#1}}

% code style
\definecolor{background_color}{RGB}{255, 255, 255}
\definecolor{string_color}    {RGB}{180, 156,   0}
\definecolor{identifier_color}{RGB}{  0,   0,   0}
\definecolor{keyword_color}   {RGB}{ 64, 100, 255}
\definecolor{comment_color}   {RGB}{  0, 117, 110}
\definecolor{number_color}    {RGB}{ 84,  84,  84}
\lstdefinestyle{15150code}{
    basicstyle=\ttfamily,
    numberstyle=\tiny\ttfamily\color{number_color},
    backgroundcolor=\color{background_color},
    stringstyle=\color{string_color},
    identifierstyle=\color{identifier_color},
    keywordstyle=\color{keyword_color},
    commentstyle=\color{comment_color},
    numbers=left,
    frame=none,
    columns=fixed,
    tabsize=2,
    breaklines=true,
    keepspaces=true,
    showstringspaces=false,
    captionpos=b,
    autogobble=true,
    mathescape=true,
    literate={~}{{$\thicksim$}}1
             {~=}{{$\eeq$}}1
}
\lstdefinelanguage{sml}{
    language=ML,
    morestring=[b]",
    morecomment=[s]{(*}{*)},
    morekeywords={
        bool, char, exn, int, real, string, unit, list, option,
        EQUAL, GREATER, LESS, NONE, SOME, nil,
        andalso, orelse, true, false, not,
        if, then, else, case, of, as,
        let, in, end, local, val, rec,
        datatype, type, exception, handle,
        fun, fn, op, raise, ref,
        structure, struct, signature, sig, functor, where,
        include, open, use, infix, infixr, o, print
    }
}
% code inline
\catcode`~=11
\newcommand{\code}[2][]{{\sloppy
\ifmmode
    \text{\lstinline[language=sml,style=15150code,#1]`#2`}
\else
    {\lstinline[language=sml,style=15150code,#1]`#2`}%
\fi}}

% code block
\lstnewenvironment{codeblock}[1][]{\lstset{language=sml,style=15150code,#1}}{}

% code imported from a file
\newcommand{\codefile}[2][]{\lstinputlisting[language=sml,style=15150code,mathescape=false,frame=single,#1]{#2}}



\begin{document}

% Sample usage:

% \code{fn x => x}

% \begin{codeblock}
%     fun fact 0 = 1
%       | fact n = n * fact (n - 1)
% \end{codeblock}

% \begin{align*}
%     \code{(fn x => 2 * x) (3 + 4)}
%       &\eeq \code{(fn x => 2 * x) 7} \tag{definition of \code{+}} \\
%       &\eeq \code{2 * 7}             \tag{function application} \\
%       &\eeq \code{14}                \tag{definition of \code{*}}
% \end{align*}

% \codefile[linerange=2-5]{path/to/file.sml}
% \codefile{path/to/file.sml} includes the whole file

\task{2.1}
% TODO: Task 2.1

\task{2.2}
% TODO: Task 2.2

\task{2.3}
% TODO: Task 2.3

\task{2.4}
% TODO: Task 2.4

\task{2.5}
% TODO: Task 2.5

\task{4.2}
\begin{codeblock}
    < Your copy-pasted code here... >
\end{codeblock}

% Or important a part of the code file:
% \codefile[linerange=1-10]{pascal.sml}
% TODO: Task 4.2

\task{6.1}
% TODO: Task 6.1

\task{6.2}
% TODO: Task 6.2

\task{6.3}
% TODO: Task 6.3

\task{7.1}
%
% This is a sample template for a nice way to format your induction proofs.
% Feel free to make changes!
%

\begin{codeblock}
fun sumEven ([] : int list ) : int = 0
  | sumEven (x::xs) =
      case x mod 2 of
        0 => x + sumEven xs
      | _ => sumEven xs
\end{codeblock}
% Alternatively, you can import directly from a code file:
% \codefile[linerange=1-10]{file.sml}

We want to show that for any \code{L : int list}, that
$$\code{sumEven L} \eeq \sum\limits_{y \in L, \; y \equiv_2 0} y$$

\begin{proof}
We proceed by structural induction on the input list, \code{L}.

\textbf{Base Case:} $\code{L} = \code{[]}$
\begin{align*}
  \code{sumEven []}
  & \eeq 0 \tag{clause 1 of \code{sumEven}} \\
  & \eeq \sum\limits_{y \in L, \; y \equiv_2 0} y \tag{vacuously true}
\end{align*}

\textbf{Inductive Step:} $\code{L} = \code{x::xs}$ where \code{x : int} is some
value and \code{xs : int list} is some value. \\
\textbf{Inductive Hypothesis:} Assume
\code{sumEven xs} $\eeq$ $\sum\limits_{y \in xs, \; y \equiv_2 0} y$. \\

\textbf{Want to Show:} \code{sumEven (x::xs)} $\eeq$
$\sum\limits_{y \in x::xs, \; y \equiv_2 0} y$.

\textbf{Case 1:} Suppose \code{x mod 2} $\eeq$ \code{0}:
\begin{align*}
        &\code{sumEven (x::xs)} \\
  \eeq\ &\code{case x mod 2 of ...}
      \tag{clause 2 of \code{sumEven}} \\
  \eeq\ &\code{x + sumEven xs}
      \tag{\code{x mod 2} $\eeq$ \code{0}} \\
  \eeq\ &\code{x +} \sum\limits_{y \in xs, \; y \equiv_2 0} y
      \tag{IH} \\
  \eeq\ &\sum\limits_{y \in x::xs, \; y \equiv_2 0} y
      \tag{\code{x} $\in$ \code{x::xs}, $x \equiv_2 0$}
\end{align*}

\textbf{Case 2:} Suppose \code{x mod 2} $\neeq$ \code{0}:
\begin{align*}
        &\code{sumEven (x::xs)} \\
  \eeq\ &\code{case x mod 2 of ...}
      \tag{clause 2 of \code{sumEven}} \\
  \eeq\ &\code{sumEven xs}
      \tag{\code{x mod 2} $\neeq$ \code{0}} \\
  \eeq\ &\sum\limits_{y \in xs, \; y \equiv_2 0} y
      \tag{IH} \\
  \eeq\ &\sum\limits_{y \in x::xs, \; y \equiv_2 0} y
      \tag{\code{x} $\in$ \code{x::xs}, \code{x mod 2} $\neeq$ \code{0}}
\end{align*}
\end{proof}
% TODO: Task 7.1

\task{8.1}
%
% This is a sample template for a nice way to format your induction proofs.
% Feel free to make changes!
%

\begin{codeblock}
fun sumEven ([] : int list ) : int = 0
  | sumEven (x::xs) =
      case x mod 2 of
        0 => x + sumEven xs
      | _ => sumEven xs
\end{codeblock}
% Alternatively, you can import directly from a code file:
% \codefile[linerange=1-10]{file.sml}

We want to show that for any \code{L : int list}, that
$$\code{sumEven L} \eeq \sum\limits_{y \in L, \; y \equiv_2 0} y$$

\begin{proof}
We proceed by structural induction on the input list, \code{L}.

\textbf{Base Case:} $\code{L} = \code{[]}$
\begin{align*}
  \code{sumEven []}
  & \eeq 0 \tag{clause 1 of \code{sumEven}} \\
  & \eeq \sum\limits_{y \in L, \; y \equiv_2 0} y \tag{vacuously true}
\end{align*}

\textbf{Inductive Step:} $\code{L} = \code{x::xs}$ where \code{x : int} is some
value and \code{xs : int list} is some value. \\
\textbf{Inductive Hypothesis:} Assume
\code{sumEven xs} $\eeq$ $\sum\limits_{y \in xs, \; y \equiv_2 0} y$. \\

\textbf{Want to Show:} \code{sumEven (x::xs)} $\eeq$
$\sum\limits_{y \in x::xs, \; y \equiv_2 0} y$.

\textbf{Case 1:} Suppose \code{x mod 2} $\eeq$ \code{0}:
\begin{align*}
        &\code{sumEven (x::xs)} \\
  \eeq\ &\code{case x mod 2 of ...}
      \tag{clause 2 of \code{sumEven}} \\
  \eeq\ &\code{x + sumEven xs}
      \tag{\code{x mod 2} $\eeq$ \code{0}} \\
  \eeq\ &\code{x +} \sum\limits_{y \in xs, \; y \equiv_2 0} y
      \tag{IH} \\
  \eeq\ &\sum\limits_{y \in x::xs, \; y \equiv_2 0} y
      \tag{\code{x} $\in$ \code{x::xs}, $x \equiv_2 0$}
\end{align*}

\textbf{Case 2:} Suppose \code{x mod 2} $\neeq$ \code{0}:
\begin{align*}
        &\code{sumEven (x::xs)} \\
  \eeq\ &\code{case x mod 2 of ...}
      \tag{clause 2 of \code{sumEven}} \\
  \eeq\ &\code{sumEven xs}
      \tag{\code{x mod 2} $\neeq$ \code{0}} \\
  \eeq\ &\sum\limits_{y \in xs, \; y \equiv_2 0} y
      \tag{IH} \\
  \eeq\ &\sum\limits_{y \in x::xs, \; y \equiv_2 0} y
      \tag{\code{x} $\in$ \code{x::xs}, \code{x mod 2} $\neeq$ \code{0}}
\end{align*}
\end{proof}
% TODO: Task 8.1

\end{document}
