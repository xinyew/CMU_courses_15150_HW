\documentclass[11pt]{article}

\usepackage{amssymb, amsmath, amsthm, amsfonts}
\usepackage{listings,xcolor,lstautogobble}
\usepackage{fancyhdr,parskip}
\usepackage[margin=0.8in]{geometry}

% homework header
\pagestyle{fancy}
\setlength{\headheight}{0.3in}
\lhead{Datatypes}
\rhead{15-150, Principles of Functional Programming}

% task environment
\newcommand{\task}[1]{\clearpage\textbf{Task #1}. \\[0.5em]}

% symbols
\newcommand{\eeq}{\ensuremath{\cong}}
\newcommand{\neeq}{\ensuremath{\ncong}}
\newcommand{\stepsTo}{\Longrightarrow}
\newcommand{\stepsToIn}[1]{\Longrightarrow^{#1}}

% code style
\definecolor{background_color}{RGB}{255, 255, 255}
\definecolor{string_color}    {RGB}{180, 156,   0}
\definecolor{identifier_color}{RGB}{  0,   0,   0}
\definecolor{keyword_color}   {RGB}{ 64, 100, 255}
\definecolor{comment_color}   {RGB}{  0, 117, 110}
\definecolor{number_color}    {RGB}{ 84,  84,  84}
\lstdefinestyle{15150code}{
    basicstyle=\ttfamily,
    numberstyle=\tiny\ttfamily\color{number_color},
    backgroundcolor=\color{background_color},
    stringstyle=\color{string_color},
    identifierstyle=\color{identifier_color},
    keywordstyle=\color{keyword_color},
    commentstyle=\color{comment_color},
    numbers=left,
    frame=none,
    columns=fixed,
    tabsize=2,
    breaklines=true,
    keepspaces=true,
    showstringspaces=false,
    captionpos=b,
    autogobble=true,
    mathescape=true,
    literate={~}{{$\thicksim$}}1
             {~=}{{$\eeq$}}1
}
\lstdefinelanguage{sml}{
    language=ML,
    morestring=[b]",
    morecomment=[s]{(*}{*)},
    morekeywords={
        bool, char, exn, int, real, string, unit, list, option,
        EQUAL, GREATER, LESS, NONE, SOME, nil,
        andalso, orelse, true, false, not,
        if, then, else, case, of, as,
        let, in, end, local, val, rec,
        datatype, type, exception, handle,
        fun, fn, op, raise, ref,
        structure, struct, signature, sig, functor, where,
        include, open, use, infix, infixr, o, print
    }
}
% code inline
\catcode`~=11
\newcommand{\code}[2][]{{\sloppy
\ifmmode
    \text{\lstinline[language=sml,style=15150code,#1]`#2`}
\else
    {\lstinline[language=sml,style=15150code,#1]`#2`}%
\fi}}

% code block
\lstnewenvironment{codeblock}[1][]{\lstset{language=sml,style=15150code,#1}}{}

% code imported from a file
\newcommand{\codefile}[2][]{\lstinputlisting[language=sml,style=15150code,mathescape=false,frame=single,#1]{#2}}


% Work+Span template
% Credit Jacob Neumann - M21
\usepackage{tikz}
\usepackage{tikz-qtree}
\usetikzlibrary{fit}
  \usetikzlibrary{trees,shapes.geometric}
    \tikzset{
    every node/.style = {align=center, inner sep=5pt, text centered},
    level 1/.style={level distance=1.6cm}
    }
\usepackage{multirow}
\renewcommand{\arraystretch}{2.5}
\definecolor{emph_color}    {RGB}{ 0,  112,  36}
\newcommand{\Step}[1]{\colorbox{emph_color}{\color{white}\textsf{#1}}}
% End of work+span template


\begin{document}

% Sample usage:

% \code{fn x => x}

% \begin{codeblock}
%     fun fact 0 = 1
%       | fact n = n * fact (n - 1)
% \end{codeblock}

% \begin{align*}
%     \code{(fn x => 2 * x) (3 + 4)}
%       &\eeq \code{(fn x => 2 * x) 7} \tag{definition of \code{+}} \\
%       &\eeq \code{2 * 7}             \tag{function application} \\
%       &\eeq \code{14}                \tag{definition of \code{*}}
% \end{align*}

% \codefile[linerange=2-5]{path/to/file.sml}
% \codefile{path/to/file.sml} includes the whole file

\task{1.1}
% TODO: Task 1.1

\task{1.2}
% TODO: Task 1.2

\task{1.3}
% TODO: Task 1.3

\task{1.4}
% TODO: Task 1.4

\task{1.5}
% TODO: Task 1.5

\task{1.6}
% TODO: Task 1.6

\task{1.7}
% TODO: Task 1.7

\task{1.8}
% TODO: Task 1.8

\task{2.1}
% TODO: Task 2.1

\task{2.2}
% TODO: Task 2.2

\task{2.3}
% TODO: Task 2.3

\task{2.4}
% TODO: Task 2.4

\task{2.5}
% TODO: Task 2.5

\task{2.6}
% TODO: Task 2.6

\task{4.5}
% Former TA and Professor Jacob Neumann made this template during M21.
%
% Using this is completely optional, if you'd rather not use it, feel free to
%   delete.

\begin{codeblock}
  (* TODO: add your find implementation *)
\end{codeblock}
\code{find} work analysis:

\begin{enumerate}
% 0. Notion of size %%%%%%%%%%%%%%%%%%%%%%%%%%%%%%%
    \item[\Step 0]
        \textbf{Notion of size: }%
    ???? % TODO

% 1. Recurrence %%%%%%%%%%%%%%%%%%%%%%%%%%%%%%%%%%%
    \item[\Step 1]\textbf{Recurrence:}
    \begin{align*}
        W(?) &= ? % TODO
    \end{align*}

% 2. Tree %%%%%%%%%%%%%%%%%%%%%%%%%%%%%%%%%%%%%%%%%
    \item[\Step 2] \textbf{Number of nodes at level $i$: }
    ???? % TODO

% 3. Work Measurements %%%%%%%%%%%%%%%%%%%%%%%%%%%%
    \item[\Step 3] \textbf{Work per node at level $i$: }
    ???? % TODO

% 4. Number of levels %%%%%%%%%%%%%%%%%%%%%%%%%%%%%
    \item[\Step 4] \textbf{Levels: }%
    ???? % TODO

% 5. Sum %%%%%%%%%%%%%%%%%%%%%%%%%%%%%%%%%%%%%%%%%%
    \item[\Step 5]\textbf{Sum:}
    \[ W(?) \approx ???? \] % TODO

% 6. Big-O %%%%%%%%%%%%%%%%%%%%%%%%%%%%%%%%%%%%%%%%
    \item[\Step 6]\textbf{Big-O: }%
    $O(??)$ % TODO

\end{enumerate}

\newpage
%%%%%%%%%%%%%%%%%%%%%%%%%%%%%%%%%%%%%%%%%%%%%%%%%%%
\code{find} span analysis:

\begin{enumerate}
% 0. Notion of size %%%%%%%%%%%%%%%%%%%%%%%%%%%%%%%
    \item[\Step 0]
        \textbf{Notion of size: }%
    ???? % TODO

% 1. Recurrence %%%%%%%%%%%%%%%%%%%%%%%%%%%%%%%%%%%
    \item[\Step 1]\textbf{Recurrence:}
    \begin{align*}
        S(?) &= ? % TODO
    \end{align*}

% 2. Tree %%%%%%%%%%%%%%%%%%%%%%%%%%%%%%%%%%%%%%%%%
    \item[\Step 2] \textbf{Number of nodes at level $i$: }
    ???? % TODO

% 3. Work Measurements %%%%%%%%%%%%%%%%%%%%%%%%%%%%
    \item[\Step 3] \textbf{Work per node at level $i$: }
    ???? % TODO

% 4. Number of levels %%%%%%%%%%%%%%%%%%%%%%%%%%%%%
    \item[\Step 4] \textbf{Levels: }%
    ???? % TODO

% 5. Sum %%%%%%%%%%%%%%%%%%%%%%%%%%%%%%%%%%%%%%%%%%
    \item[\Step 5]\textbf{Sum:}
    \[ S(?) \approx ???? \] % TODO

% 6. Big-O %%%%%%%%%%%%%%%%%%%%%%%%%%%%%%%%%%%%%%%%
    \item[\Step 6]\textbf{Big-O: }%
    $O(??)$ % TODO

\end{enumerate}

\newpage
\begin{codeblock}
  (* TODO: add your follow implementation *)
\end{codeblock}
\code{follow} work analysis:

\begin{enumerate}
% 0. Notion of size %%%%%%%%%%%%%%%%%%%%%%%%%%%%%%%
    \item[\Step 0]
        \textbf{Notion of size: }%
    ???? % TODO

% 1. Recurrence %%%%%%%%%%%%%%%%%%%%%%%%%%%%%%%%%%%
    \item[\Step 1]\textbf{Recurrence:}
    \begin{align*}
        W(?) &= ? % TODO
    \end{align*}

% 2. Tree %%%%%%%%%%%%%%%%%%%%%%%%%%%%%%%%%%%%%%%%%
    \item[\Step 2] \textbf{Number of nodes at level $i$: }
    ???? % TODO

% 3. Work Measurements %%%%%%%%%%%%%%%%%%%%%%%%%%%%
    \item[\Step 3] \textbf{Work per node at level $i$: }
    ???? % TODO

% 4. Number of levels %%%%%%%%%%%%%%%%%%%%%%%%%%%%%
    \item[\Step 4] \textbf{Levels: }%
    ???? % TODO

% 5. Sum %%%%%%%%%%%%%%%%%%%%%%%%%%%%%%%%%%%%%%%%%%
    \item[\Step 5]\textbf{Sum:}
    \[ W(?) \approx ???? \] % TODO

% 6. Big-O %%%%%%%%%%%%%%%%%%%%%%%%%%%%%%%%%%%%%%%%
    \item[\Step 6]\textbf{Big-O: }%
    $O(??)$ % TODO

\end{enumerate}

\newpage
%%%%%%%%%%%%%%%%%%%%%%%%%%%%%%%%%%%%%%%%%%%%%%%%%%%
\code{follow} span analysis:

\begin{enumerate}
% 0. Notion of size %%%%%%%%%%%%%%%%%%%%%%%%%%%%%%%
    \item[\Step 0]
        \textbf{Notion of size: }%
    ???? % TODO

% 1. Recurrence %%%%%%%%%%%%%%%%%%%%%%%%%%%%%%%%%%%
    \item[\Step 1]\textbf{Recurrence:}
    \begin{align*}
        S(?) &= ? % TODO
    \end{align*}

% 2. Tree %%%%%%%%%%%%%%%%%%%%%%%%%%%%%%%%%%%%%%%%%
    \item[\Step 2] \textbf{Number of nodes at level $i$: }
    ???? % TODO

% 3. Work Measurements %%%%%%%%%%%%%%%%%%%%%%%%%%%%
    \item[\Step 3] \textbf{Work per node at level $i$: }
    ???? % TODO

% 4. Number of levels %%%%%%%%%%%%%%%%%%%%%%%%%%%%%
    \item[\Step 4] \textbf{Levels: }%
    ???? % TODO

% 5. Sum %%%%%%%%%%%%%%%%%%%%%%%%%%%%%%%%%%%%%%%%%%
    \item[\Step 5]\textbf{Sum:}
    \[ W(?) \approx ???? \] % TODO

% 6. Big-O %%%%%%%%%%%%%%%%%%%%%%%%%%%%%%%%%%%%%%%%
    \item[\Step 6]\textbf{Big-O: }%
    $O(??)$ % TODO

\end{enumerate}


\newpage
\begin{codeblock}
  (* TODO: add your lca implementation *)
\end{codeblock}
\code{lca} work analysis:

\begin{enumerate}
% 0. Notion of size %%%%%%%%%%%%%%%%%%%%%%%%%%%%%%%
    \item[\Step 0]
        \textbf{Notion of size: }%
    ???? % TODO

% 1. Recurrence %%%%%%%%%%%%%%%%%%%%%%%%%%%%%%%%%%%
    \item[\Step 1]\textbf{Recurrence:}
    \begin{align*}
        W(?) &= ? % TODO
    \end{align*}

% 2. Tree %%%%%%%%%%%%%%%%%%%%%%%%%%%%%%%%%%%%%%%%%
    \item[\Step 2] \textbf{Number of nodes at level $i$: }
    ???? % TODO

% 3. Work Measurements %%%%%%%%%%%%%%%%%%%%%%%%%%%%
    \item[\Step 3] \textbf{Work per node at level $i$: }
    ???? % TODO

% 4. Number of levels %%%%%%%%%%%%%%%%%%%%%%%%%%%%%
    \item[\Step 4] \textbf{Levels: }%
    ???? % TODO

% 5. Sum %%%%%%%%%%%%%%%%%%%%%%%%%%%%%%%%%%%%%%%%%%
    \item[\Step 5]\textbf{Sum:}
    \[ W(?) \approx ???? \] % TODO

% 6. Big-O %%%%%%%%%%%%%%%%%%%%%%%%%%%%%%%%%%%%%%%%
    \item[\Step 6]\textbf{Big-O: }%
    $O(??)$ % TODO

\end{enumerate}

\newpage
%%%%%%%%%%%%%%%%%%%%%%%%%%%%%%%%%%%%%%%%%%%%%%%%%%%
\code{lca} span analysis:

\begin{enumerate}
% 0. Notion of size %%%%%%%%%%%%%%%%%%%%%%%%%%%%%%%
    \item[\Step 0]
        \textbf{Notion of size: }%
    ???? % TODO

% 1. Recurrence %%%%%%%%%%%%%%%%%%%%%%%%%%%%%%%%%%%
    \item[\Step 1]\textbf{Recurrence:}
    \begin{align*}
        S(?) &= ? % TODO
    \end{align*}

% 2. Tree %%%%%%%%%%%%%%%%%%%%%%%%%%%%%%%%%%%%%%%%%
    \item[\Step 2] \textbf{Number of nodes at level $i$: }
    ???? % TODO

% 3. Work Measurements %%%%%%%%%%%%%%%%%%%%%%%%%%%%
    \item[\Step 3] \textbf{Work per node at level $i$: }
    ???? % TODO

% 4. Number of levels %%%%%%%%%%%%%%%%%%%%%%%%%%%%%
    \item[\Step 4] \textbf{Levels: }%
    ???? % TODO

% 5. Sum %%%%%%%%%%%%%%%%%%%%%%%%%%%%%%%%%%%%%%%%%%
    \item[\Step 5]\textbf{Sum:}
    \[ S(?) \approx ???? \] % TODO

% 6. Big-O %%%%%%%%%%%%%%%%%%%%%%%%%%%%%%%%%%%%%%%%
    \item[\Step 6]\textbf{Big-O: }%
    $O(??)$ % TODO

\end{enumerate}


% TODO: Task 4.5

\task{5.2}
% TODO: Task 5.2

\task{6.2}
% TODO: Task 6.2

\task{6.3}
% TODO: Task 6.3

\task{6.4}
\begin{codeblock}
  (* TODO: add your eval implementation *)
\end{codeblock}
% TODO: Task 6.4

\task{6.5}
\begin{codeblock}
  (* TODO: add your eval implementation *)
\end{codeblock}
% TODO: Task 6.5

\end{document}
