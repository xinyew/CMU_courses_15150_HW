\documentclass[11pt]{article}

\usepackage{amssymb, amsmath, amsthm, amsfonts}
\usepackage{listings,xcolor,lstautogobble}
\usepackage{fancyhdr,parskip}
\usepackage[margin=0.8in]{geometry}

% homework header
\pagestyle{fancy}
\setlength{\headheight}{0.3in}
\lhead{Basics}
\rhead{15-150, Principles of Functional Programming}

% task environment
\newcommand{\task}[1]{\clearpage\textbf{Task #1}. \\[0.5em]}

% symbols
\newcommand{\eeq}{\ensuremath{\cong}}
\newcommand{\neeq}{\ensuremath{\ncong}}
\newcommand{\stepsTo}{\Longrightarrow}
\newcommand{\stepsToIn}[1]{\Longrightarrow^{#1}}

% code style
\definecolor{background_color}{RGB}{255, 255, 255}
\definecolor{string_color}    {RGB}{180, 156,   0}
\definecolor{identifier_color}{RGB}{  0,   0,   0}
\definecolor{keyword_color}   {RGB}{ 64, 100, 255}
\definecolor{comment_color}   {RGB}{  0, 117, 110}
\definecolor{number_color}    {RGB}{ 84,  84,  84}
\lstdefinestyle{15150code}{
    basicstyle=\ttfamily,
    numberstyle=\tiny\ttfamily\color{number_color},
    backgroundcolor=\color{background_color},
    stringstyle=\color{string_color},
    identifierstyle=\color{identifier_color},
    keywordstyle=\color{keyword_color},
    commentstyle=\color{comment_color},
    numbers=left,
    frame=none,
    columns=fixed,
    tabsize=2,
    breaklines=true,
    keepspaces=true,
    showstringspaces=false,
    captionpos=b,
    autogobble=true,
    mathescape=true,
    literate={~}{{$\thicksim$}}1
             {~=}{{$\eeq$}}1
}
\lstdefinelanguage{sml}{
    language=ML,
    morestring=[b]",
    morecomment=[s]{(*}{*)},
    morekeywords={
        bool, char, exn, int, real, string, unit, list, option,
        EQUAL, GREATER, LESS, NONE, SOME, nil,
        andalso, orelse, true, false, not,
        if, then, else, case, of, as,
        let, in, end, local, val, rec,
        datatype, type, exception, handle,
        fun, fn, op, raise, ref,
        structure, struct, signature, sig, functor, where,
        include, open, use, infix, infixr, o, print
    }
}
% code inline
\catcode`~=11
\newcommand{\code}[2][]{{\sloppy
\ifmmode
    \text{\lstinline[language=sml,style=15150code,#1]`#2`}
\else
    {\lstinline[language=sml,style=15150code,#1]`#2`}%
\fi}}

% code block
\lstnewenvironment{codeblock}[1][]{\lstset{language=sml,style=15150code,#1}}{}

% code imported from a file
\newcommand{\codefile}[2][]{\lstinputlisting[language=sml,style=15150code,mathescape=false,frame=single,#1]{#2}}



\begin{document}

% Sample usage:

\code{fn x => x}

\begin{codeblock}
    fun fact 0 = 1
      | fact n = n * fact (n - 1)
\end{codeblock}

\begin{align*}
    \code{(fn x => 2 * x) (3 + 4)}
      &\eeq \code{(fn x => 2 * x) 7} \tag{definition of \code{+}} \\
      &\eeq \code{2 * 7}             \tag{function application} \\
      &\eeq \code{14}                \tag{definition of \code{*}}
\end{align*}

\begin{align*}
    \code{(fn x => 2 * x) (3 + 4)}
      &\eeq \code{(fn x => 2 * x) 7} \tag{definition of \code{+}} \\
      &\eeq \code{2 * 7}             \tag{function application} \\
      &\eeq \code{14}                \tag{definition of \code{*}}
\end{align*}


% \codefile[linerange=2-5]{path/to/file.sml}
% \codefile{path/to/file.sml} includes the whole file

\task{2.1}
% TODO: Task 2.1
\begin{codeblock}
\end{codeblock}

\task{2.2}
% TODO: Task 2.2
\begin{codeblock}
    fun fact (0 : int) : int = 1
      | fact (n : int) : int = n * fact (n - 1)
\end{codeblock}
We can see by our type annotations that this function has type \code{int -> int}!

\task{2.3}
% TODO: Task 2.3
\code{fact : int -> int}

\task{2.4}
% TODO: Task 2.4

\task{3.1}
% TODO: Task 3.1

\task{3.2}
% TODO: Task 3.2

\task{3.3}
% TODO: Task 3.3

\task{3.4}
% TODO: Task 3.4

\task{3.5}
% TODO: Task 3.5

\task{3.6}
% TODO: Task 3.6

\task{3.7}
% TODO: Task 3.7

\task{3.8}
% TODO: Task 3.8

\task{3.9}
% TODO: Task 3.9

\task{3.10}
% TODO: Task 3.10

\task{4.1}
% TODO: Task 4.1

\task{4.2}
% TODO: Task 4.2

\task{4.3}
% TODO: Task 4.3

\task{4.4}
% TODO: Task 4.4

\task{4.5}
% TODO: Task 4.5

\task{5.1}
% TODO: Task 5.1

\task{5.2}
% TODO: Task 5.2

\task{5.3}
% TODO: Task 5.3

\task{6.1}
% TODO: Task 6.1

\task{6.3}
% TODO: Task 6.3

\task{6.5}
% TODO: Task 6.5

\task{6.7}
% TODO: Task 6.7

\task{6.9}
% TODO: Task 6.9

\task{7.1}
% TODO: Task 7.1

\task{7.2}
% TODO: Task 7.2

\task{7.3}
% TODO: Task 7.3

\end{document}
